\PassOptionsToPackage{pdfpagelabels=false}{hyperref} 
\documentclass[useAMS,usenatbib, usedcolumn]{mnras}
\pdfoutput=1 %for arxiv 
\usepackage{graphicx}
\usepackage{color}
\usepackage{amsmath,amssymb,amstext}
\usepackage[T1]{fontenc}
\usepackage{mathptmx,txfonts}
\usepackage[figure,figure*]{hypcap}
\usepackage[most]{tcolorbox}
\usepackage{ae,aecompl}
\usepackage[normalem]{ulem} % Added by MS for \sout -> not required for final version
\input{hyperlink-year-only-natbib-patch}

\hypersetup{
    pdfauthor={Victor F. Calderon},
    pdftitle={Main title},
    pdfkeywords={cosmology: observations --- cosmology: large-scale structure 
            of Universe --- galaxies: groups --- galaxies:clusters
            --- methods: statistical, machine learning},
    colorlinks=true,
    citecolor=blue,
    linkcolor=magenta,
    % filecolor=magenta,
    urlcolor=cyan}

%%%%%%%%%%%%%%%%%%%%%%%%%%%%%%%%%%%%%%%%%%%
%%%%%%%%%% ------ Modules ------ %%%%%%%%%%
%%%%%%%%%%%%%%%%%%%%%%%%%%%%%%%%%%%%%%%%%%%
\usepackage[utf8]{inputenc}
\usepackage[english]{babel}
\usepackage{array}
\usepackage{afterpage}
\usepackage{color}
\usepackage{graphicx}
\usepackage{amsmath, amssymb}
\usepackage{soul,xcolor}
\usepackage{epstopdf}
\usepackage{units}
\usepackage{relsize}
\usepackage{enumitem}
\usepackage{url}
\usepackage{appendix}
\usepackage{footnote}
\usepackage{xspace}
\usepackage{float}

\usepackage{booktabs}
\usepackage{tabularx}

\usepackage{color}
\usepackage{graphicx}
\usepackage{epstopdf}
\usepackage{enumitem}
\usepackage{xspace}
\usepackage{float}
\usepackage{booktabs}
\usepackage{tabularx}

\usepackage{relsize}
\usepackage{array}
\usepackage{afterpage}
\usepackage{units}
\usepackage{url}
\usepackage{appendix}
\usepackage{footnote}
\usepackage{natbib}

%%% Graphics path
\graphicspath{{./Figures/}}

\newcolumntype{L}[1]{>{\raggedright\let\newline\\\arraybackslash\hspace{0pt}}p{#1}}
\newcolumntype{C}[1]{>{\centering\let\newline\\\arraybackslash\hspace{0pt}}p{#1}}
\newcolumntype{R}[1]{>{\raggedleft\let\newline\\\arraybackslash\hspace{0pt}}p{#1}}

%%%%%%%%%%%%%%%%%%%%%%%%%%%%%%%%%%%%%%%%%%%%
%%%%%%%%%% ------ Commands ------ %%%%%%%%%%
%%%%%%%%%%%%%%%%%%%%%%%%%%%%%%%%%%%%%%%%%%%%
\setstcolor{red}
\definecolor{change}{RGB}{255,3,3}
\def\change {{\color{change}} }
\def\starfpy {\textsc{StarfPy}}
\newcommand{\refsec}[1]{$\S$\ref{#1}}

\newcommand{\msun}{h^{-1}M_\odot}
%\newcommand{\msunh}{\mbox{$h^{-1} \textrm{M}_{\odot}$}}
% \newcommand{\kpch}{\mbox{$\textrm{kpc}/h$}}
\newcommand{\hkpc}{\mbox{$h^{-1}\textrm{kpc}$}}
\newcommand{\hmpc}{\mbox{$h^{-1}\textrm{Mpc}$} }
\newcommand{\kmsMpc}{\mbox{$\textrm{km}\ \textrm{s}^{-1}\ \textrm{Mpc}^{-1}$}}
% \newcommand{\mpch}{\mbox{$\textrm{kpc}/h$}}
% \newcommand{\kms}{$\textrm{km}s^{-1}$}
\newcommand{\hmpcthreeinv}{\mbox{$h^{3} \textrm{Mpc}^{-3}$}}
\newcommand{\mstar}{\mbox{$\textrm{M}_{\star}$}}
\newcommand{\gr}{\ensuremath{(g-r)}\xspace}
\newcommand{\logssfr}{$\log\ \textrm{sSFR}$}
\newcommand{\xirp}{$\xi(r_{p})$}
\newcommand{\rband}{\textit{r}-band}
\newcommand{\gband}{\textit{g}-band }
\newcommand{\bfof}{\texttt{berlind-fof}\xspace}
\newcommand{\MD}[1]{{\texttt Mr#1-SDSS}\xspace}
\newcommand{\MM}[1]{{\texttt Mr#1-Mock}\xspace}
\newcommand{\MR}[1]{{\texttt Mr#1}\xspace}
\newcommand{\Mr}[1]{{Mr#1}\xspace}
\newcommand{\mgroup}{\ensuremath{M_\textrm{group}}\xspace}
\newcommand{\MPA}{MPA-JHU }
\newcommand{\mcf}{\ensuremath{\mathcal{M}(r_{p})}\xspace}
\newcommand{\mcft}{\texorpdfstring{\mcf}{MCF}}
\newcommand{\sersic}{{S\'ersic}\xspace}
% \newcommand{\onehn}{1-halo}
% \newcommand{\twohn}{2-halo}
%
\newcommand{\failval}{-999 }
\newcommand{\ssfr}{{sSFR}\xspace}
\newcommand{\sfr}{{SFR}\xspace}
\newcommand{\lcdm}{{\ensuremath{\Lambda}CDM}\xspace}
\newcommand{\figwidth}{0.47}
\newcommand{\figheight}{0.80}
\newcommand{\catsurl}{\url{https://github.com/vcalderon2009/SDSS_Groups_ML}}
\newcommand{\dsum}{\displaystyle \sum\limits}

% Please keep new commands to a minimum, and use \newcommand not \def to avoid
% overwriting existing commands. Example:
%\newcommand{\pcm}{\,cm$^{-2}$}	% per cm-squared
\input{macros_dc.sty}

% for comments:
%% MS - had to define the color for my edits
\definecolor{orange}{rgb}{1,0.5,0}
\setlength{\marginparwidth}{1.2in}
\let\oldmarginpar\marginpar
\renewcommand\marginpar[1]{\-\oldmarginpar[\raggedleft\footnotesize #1]%
{\raggedright\footnotesize #1}}

%%% Here are the things that make collaborative effort easier
\newcommand{\todo}[1]{\marginpar{\color{red}TODO}{\color{red}#1}}
\newcommand{\ab}[1]{{\color{blue}{\bf AB:} #1}}
\newcommand{\vc}[1]{{\color{cyan}{\bf VC:} #1}}
%%%%%%%%%%%%%%%%%%%%%%%%%%%%%%%%%%%%%%%%%%%%%%%%%%

%%%% Citing previous papers
\defcitealias{Calderon2018}{C18}
\defcitealias{Zu2016a}{Zu16}


%%%%%%%%%%%%%%%%%%% TITLE PAGE %%%%%%%%%%%%%%%%%%%

% Title
\title{Prediction of Cluster and Group masses in SDSS DR7 via a machine learning approach}
% Authors
\author[Calderon et al.]{%
Victor F. Calderon$^{1}$\thanks{E-mail:  \mailto{victor.calderon@vanderbilt.edu}},
Andreas A. Berlind$^{1}$\vspace*{0.2em} \\ 
% List of Institutions
$^{1}$Department of Physics and Astronomy, Vanderbilt University, Nashville, TN 37235, USA
}

% These dates will be filled out by the publisher
\date{Accepted XXX. Received YYY; in original form ZZZ}

% Enter the current year, for the copyright statements etc.
\pubyear{2018}

% Don't change these lines
\begin{document}
\label{firstpage}
\pagerange{\pageref{firstpage}--\pageref{lastpage}}
\maketitle

\begin{abstract}
We present a machine learning approach (ML) for the prediction of 
Cluster and Group masses with an improved performance over
using conventional methods. We make use of synthetic catalogues
that aim to emulate the clustering of galaxies and joint-distributions
of properties of galaxis in the real Universe in order to train
three ML algorithms to correctly predict cluster and galaxy group masses.
\end{abstract}

\begin{keywords}
cosmology: observations --- cosmology: large-scale structure 
            of Universe --- galaxies: groups --- galaxies:clusters
            --- methods: statistical
\end{keywords}

%%%%%%%%%%%%%%%%%%%%%%%%%%%%%%%%%%%%%%%%%%%%%%%%%%%%%%%%%%%%%%%%%%
%%%%%%%%%%%%%%% ------     Introduction     ------ %%%%%%%%%%%%%%%
%%%%%%%%%%%%%%%%%%%%%%%%%%%%%%%%%%%%%%%%%%%%%%%%%%%%%%%%%%%%%%%%%%

\section{Introduction}
\label{sec:intro}

\begin{itemize}[leftmargin=0.5\parindent, labelsep=0.5\parindent]
    \item
    Current literature on previous analyses that have used ML to
    determine masses of galaxy groups and galaxy clusters.

    \item
    How has ML been used before to estimate properties of
    galaxy groups. See `Paul2015'. Also see \citet{Ntampaka2015}.

\end{itemize}

%%%%%%%%%%%%%%%%%%%%%%%%%%%%%%%%%%%%%%%%%%%%%%%%%%%%%%%%%%%%%%%%%%
%%%%%%%%%%%%%%%%%%% ------     Data     ------ %%%%%%%%%%%%%%%%%%%
%%%%%%%%%%%%%%%%%%%%%%%%%%%%%%%%%%%%%%%%%%%%%%%%%%%%%%%%%%%%%%%%%%

%%% ============================================ %%%
\section{Data and Methods}
\label{sec:data_meas}

\begin{itemize}[leftmargin=0.5\parindent, labelsep=0.5\parindent]
    \item
    Summarize the different subsections of the \textit{data} main section.

    \item
    Introduce the catalogues presented in \citet{Calderon2018}.
    Especifically \S 2.1 - 2.3.

    \item
    Characterize the different features that we use as part of the ML
    training. Describe in depth the different \textit{features} used.

\end{itemize}

In this section, we present the datasets used throughout this analysis,
and introdue the main machine learning algorithms and statistical
methods that we use to estimate the masses of galaxy groups and galaxy
clusters. In \refsec{subsec:SDSS_galaxy_catalogues}, we briefly describe
the SDSS galaxy sample and mock synthethic galaxy catalogues that we use,
along with the parameters that are included in these catalogue. In
\refsec{subsec:galprop_features}, we introduce the different \textit{features}
that we use for training our ML predictors, and provide a guide on
how these are calculated. Finally, in \refsec{subsec:ml_algorithms} we
provide a brief overview of the different algorithms that we use in
this analysis, as well as the default tuning parameters used by each
algorithm.


%%% ============================================ %%%
%%% --- Galaxy sample and mock galaxy catalogues %%%
%%% ============================================ %%%
\subsection{SDSS Galaxy Sample and Mock Galaxy Catalogues}
\label{subsec:SDSS_galaxy_catalogues}

\begin{itemize}[leftmargin=0.5\parindent, labelsep=0.5\parindent]
    \item
    Introduce the catalogues used in this analysis
    
    \item
    Reference to \citet{Calderon2018} about the mock catalogues 
    used in this analysis.
    
    \item
    Briefly mention how galaxy properties are assigned to mock
    galaxies
    
    \item
    Also mention the \textit{group-finding} step taken for each
    galaxy group.
\end{itemize}

%%% ============================================ %%%
%%% --- SDSS Galaxy Sample                       %%%
%%% ============================================ %%%
\subsubsection{SDSS Galaxy Sample}
\label{subsubsec:NYU_DR7}


%%% ============================================ %%%
%%% --- Mock Galaxy catalogues                   %%%
%%% ============================================ %%%
\subsubsection{Mock Galaxy Catalogues}
\label{subsubsec:Mock_Catls}


%%% ============================================ %%%
%%% --- Galaxy properties as 'features' ---      %%%
%%% ============================================ %%%
\subsection{Galaxy properties as \textit{features}}
\label{subsec:galprop_features}

\begin{itemize}[leftmargin=0.5\parindent, labelsep=0.5\parindent]
    \item
    In this section, I summarize the different \textit{features} used
    during the training of the algorithms.

\end{itemize}

%%% ============================================ %%%
%%% --- Machine Learning algorithms ---          %%%
%%% ============================================ %%%
\subsection{Machine Learning algorithms}
\label{subsec:ml_algorithms}

\begin{itemize}[leftmargin=0.5\parindent, labelsep=0.5\parindent]
    \item
    Talk about the different algorithms, i.e. Random Forest
    \citet{Breiman2001}, XGBoost \citep{Chen2016}.

    \item Have a separate section for each algorithm.
\end{itemize}


%%% ============================================ %%%
%%% --- General Overview section ---             %%%
%%% ============================================ %%%
\subsubsection{General Overview}
\label{subsub:ml_gen_overview}



%%% ============================================ %%%
%%% --- XGBoost section ---                      %%%
%%% ============================================ %%%
\subsubsection{XGBoost}
\label{subsub:ml_xgboost}



%%% ============================================ %%%
%%% --- Random Forests section ---               %%%
%%% ============================================ %%%
\subsubsection{Random Forest}
\label{subsub:ml_random_forest}



%%% ============================================ %%%
%%% --- Neural network section ---               %%%
%%% ============================================ %%%
\subsubsection{Neural network}
\label{subsub:ml_neural_network}





%%% ============================================ %%%
%%% --- Results ---                              %%%
%%% ============================================ %%%
\section{Results}
\label{sec:results}

\begin{itemize}[leftmargin=0.5\parindent, labelsep=0.5\parindent]
    \item
    Include reference to \citet{Ivezic2014}

    \item
    Section on the different metrics used to evaluate how well the
    algorithms are performing.

    \item
    Section on the different types of predictions for the different
    types of models, i.e. different \textit{velocity bias} ($\sigma_{v}$)
    and different \textit{HOD} models at fixed $\sigma_{v}$.
\end{itemize}

In this section, we present the results of the mass predictions for
galaxy groups and galaxy clusters of SDSS DR7.


%%% ============================================ %%%
%%% --- Results --- Fixed HOD and Velocity Bias  %%%
%%% ============================================ %%%
\subsection{Fixed HOD and Velocity bias}
\label{subsec:fixed_hod_dv}


%%% ============================================ %%%
%%% --- Results --- Varying HOD models  %%%
%%% ============================================ %%%
\subsection{Variations of HOD and Velocity Bias}
\label{subsec:hod_dv_variations}


%%% ============================================ %%%
%%% --- Results --- Varying HOD models  %%%
%%% ============================================ %%%
\subsubsection{Varying HOD models}
\label{subsubsec:fixed_dv_only}



%%% ============================================ %%%
%%% --- Results --- Varying HOD models  %%%
%%% ============================================ %%%
\subsubsection{Varying Velocity bias models}
\label{subsubsec:fixed_HOD_only}








%%% ============================================ %%%
\section{Summary and Discussion}\label{sec:summary_discussion}

\begin{itemize}[leftmargin=0.5\parindent, labelsep=0.5\parindent]
    \item
    Summarize all of the findings.

    \item
    Mention the website, at which the catalogues can be found.
    
\end{itemize}





%%% ============================================ %%%
\section{Acknowledgements}\label{sec:acknowledments}

The mock catalogues used in this paper were produced by the LasDamas project 
(\url{http://lss.phy.vanderbilt.edu/lasdamas/}); we thank NSF XSEDE for 
providing the computational resources for LasDamas. Some of the 
computational facilities used in this project were provided by the 
Vanderbilt Advanced Computing Center for Research and Education (ACCRE). 
This project has been supported by the National Science Foundation (NSF) 
through a Career Award (AST-1151650). 
Parts of this research were conducted by the Australian Research Council 
Centre of Excellence for All Sky Astrophysics in 3 Dimensions (ASTRO 3D), 
through project number CE170100013.
This research has made use of 
NASA's Astrophysics Data System. This work made use of the IPython package 
\citep{Perez2007}, Scikit-learn \citep{McKinney2010}, SciPy 
\citep{jones_scipy_2001}, matplotlib, a Python library for publication 
quality graphics \citep{Hunter:2007}, Astropy, a community-developed 
core Python package for Astronomy \citep{AstropyCollaboration2013}, and 
NumPy \citep{VanDerWalt2011}. Funding for the SDSS and SDSS-II has been 
provided by the Alfred P. Sloan Foundation, the Participating Institutions, 
the National Science Foundation, the U.S. Department of Energy, 
the National Aeronautics and Space Administration, the 
Japanese Monbukagakusho, the Max Planck Society, and the Higher Education 
Funding Council for England. The SDSS Web Site is http://www.sdss.org/. 
The SDSS is managed by the Astrophysical Research Consortium for the 
Participating Institutions. The Participating Institutions are the 
American Museum of Natural History, Astrophysical Institute Potsdam, 
University of Basel, University of Cambridge, Case Western Reserve 
University, University of Chicago, Drexel University, Fermilab, the 
Institute for Advanced Study, the Japan Participation Group, 
Johns Hopkins University, the Joint Institute for Nuclear Astrophysics, 
the Kavli Institute for Particle Astrophysics and Cosmology, 
the Korean Scientist Group, the Chinese Academy of Sciences (LAMOST), 
Los Alamos National Laboratory, the Max-Planck-Institute for Astronomy (MPIA), 
the Max-Planck-Institute for Astrophysics (MPA), 
New Mexico State University, Ohio State University, 
University of Pittsburgh, University of Portsmouth, Princeton University, 
the United States Naval Observatory, and the University of Washington. 
These acknowledgements were compiled using the Astronomy Acknowledgement 
Generator.

%%%%%%%%%%%%%%%%%%%%%%%%%%%%%%%%%%%%%%%%%%%%%%%%%%%%%%%%%%%%%%%%%%
%%%%%%%%%%%%%% ------     Bibliography     ------ %%%%%%%%%%%%%%%%
%%%%%%%%%%%%%%%%%%%%%%%%%%%%%%%%%%%%%%%%%%%%%%%%%%%%%%%%%%%%%%%%%%
% \bibliographystyle{mn2e}
\bibliographystyle{mnras}
\bibliography{Mendeley}

% Don't change these lines
\bsp	% typesetting comment
\label{lastpage}
\end{document}
