%%
%%
%%%%%%%%%%%%%%%%%%%%%%%%%%%%%%%%%%%%%%%%%%%%%%%%%%%%%%%%%%%%%%%%%%%%%%%%%%%%%%
%%
%% Main MNRAS template for Cookicutter project
%%
%% Section -- ACKWNOLEDGEMENTS
%%
%% Author: Victor F. Calderon (http://vcalderon.me)
%%
%%%%%%%%%%%%%%%%%%%%%%%%%%%%%%%%%%%%%%%%%%%%%%%%%%%%%%%%%%%%%%%%%%%%%%%%%%%%%%
%%
%% In here, you will add the 'abstract' of your paper

\section{Acknowledgements}
\label{sec:acknowledgements}

This is where the acknowledgements go.
It is useful to use
\textbf{Acknowledgement Generator} (\https{astrofrog.github.io/acknowledgment-generator/})
to create your acknowledgements. This lets you add the bibitem entried for
each of the packages used in the analysis.

The mock catalogues used in this paper were produced by the LasDamas project 
(\url{http://lss.phy.vanderbilt.edu/lasdamas/}); we thank NSF XSEDE for 
providing the computational resources for LasDamas. Some of the 
computational facilities used in this project were provided by the 
Vanderbilt Advanced Computing Center for Research and Education (ACCRE). 
This project has been supported by the National Science Foundation (NSF) 
through a Career Award (AST-1151650). 
Parts of this research were conducted by the Australian Research Council 
Centre of Excellence for All Sky Astrophysics in 3 Dimensions (ASTRO 3D), 
through project number CE170100013.
This research has made use of 
NASA's Astrophysics Data System. This work made use of the IPython package 
\citep{Perez2007}, Scikit-learn \citep{McKinney2010}, SciPy 
\citep{jones_scipy_2001}, matplotlib, a Python library for publication 
quality graphics \citep{Hunter:2007}, Astropy, a community-developed 
core Python package for Astronomy \citep{AstropyCollaboration2013}, and 
NumPy \citep{VanDerWalt2011}. Funding for the SDSS and SDSS-II has been 
provided by the Alfred P. Sloan Foundation, the Participating Institutions, 
the National Science Foundation, the U.S. Department of Energy, 
the National Aeronautics and Space Administration, the 
Japanese Monbukagakusho, the Max Planck Society, and the Higher Education 
Funding Council for England. The SDSS Web Site is http://www.sdss.org/. 
The SDSS is managed by the Astrophysical Research Consortium for the 
Participating Institutions. The Participating Institutions are the 
American Museum of Natural History, Astrophysical Institute Potsdam, 
University of Basel, University of Cambridge, Case Western Reserve 
University, University of Chicago, Drexel University, Fermilab, the 
Institute for Advanced Study, the Japan Participation Group, 
Johns Hopkins University, the Joint Institute for Nuclear Astrophysics, 
the Kavli Institute for Particle Astrophysics and Cosmology, 
the Korean Scientist Group, the Chinese Academy of Sciences (LAMOST), 
Los Alamos National Laboratory, the Max-Planck-Institute for Astronomy (MPIA), 
the Max-Planck-Institute for Astrophysics (MPA), 
New Mexico State University, Ohio State University, 
University of Pittsburgh, University of Portsmouth, Princeton University, 
the United States Naval Observatory, and the University of Washington. 
These acknowledgements were compiled using the Astronomy Acknowledgement 
Generator.